% Meta-Informationen -----------------------------------------------------------
%   Definition von globalen Parametern, die im gesamten Dokument verwendet
%   werden k�nnen (z.B auf dem Deckblatt etc.).
%
%   ACHTUNG: Wenn die Texte Umlaute oder ein Esszet enthalten, muss der folgende
%            Befehl bereits an dieser Stelle aktiviert werden:
%            \usepackage[latin1]{inputenc}
% ------------------------------------------------------------------------------
\newcommand{\titel}{Mediaplayer {\Bezeichnung} f�r den Raspberry Pi Modell B+}
\newcommand{\untertitel}{}%{TODO: und hier kommt der Untertitel}
\newcommand{\Bezeichnung}{YAMuPlay}
\newcommand{\BezeichnungLang}{Yet Another Music Player}
\newcommand{\Version}{V0.1}
\newcommand{\Dokumentart}{D O K U M E N T A T I O N}
\newcommand{\autor}{schlizb�da}
\newcommand{\jahr}{2016}
\newcommand{\bauwong}{Bauwong n.e.V.}
\newcommand{\logo}{bauwong.pdf}

\newcommand{\WandermontageDoppelt}{\textit{10125 Montagelinie HMI}}
\newcommand{\DienststelleME}{I IA CE CP MF GWA ME 2 M-720}
\newcommand{\SimaticNetSoftware}{SIMATIC NET PC-Software V12+SP2}

% verwendete Hardware
\newcommand{\RPi}{Raspberry Pi}
\newcommand{\Ligawo}{Ligawo 6518725 HDMI Extractor}
% http://www.amazon.de/Ligawo-%C2%AE-HDMI-Audio-Extractor/dp/B00CAQN0CM/ref=cm_cr_pr_product_top?ie=UTF8&tag=httpwwwforumr-21  Variante C: RCA (Cinch)
% http://www.amazon.de/dp/B00ADJIVB8 (RPi-Forum)

% verwendete Software
\newcommand{\matchboxKeyboard}{matchbox-keyboard}



%Steuerelemente von Software:
\newcommand{\prompt}[1]{\Code{\textit{#1}}}
\newcommand{\filenam}[1]{\Code{#1}}

\newcommand{\button}[1]{\Code{[{#1}]}}
\newcommand{\menuitem}[1]{\textbf{\textit{"{#1}"}}}
\newcommand{\checkbox}[1]{\textbf{\textit{"{#1}"}}}

\newcommand{\Verein}[1]{\textit{#1}}


%Smileys:
\newcommand{\smiley}[1]{\includegraphics[width=0.3cm]{Bilder/smileys/{#1}}}
